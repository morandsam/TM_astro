% !TeX encoding = UTF-8
% !TeX program = xelatex
% !TeX spellcheck = fr
% !TeX root = tm_astro_main.tex


\chapter*{Conclusion}

\addcontentsline{toc}{chapter}{Conclusion}

\vfill

Nous avons tout au long de ce travail étudié le cycle de vie des étoiles, en posant tout d'abord une base théorique concernant la structure des étoiles, puis en traitant de manière plus qualitative de la vie des étoiles et plus particulièrement de leur mort. L'E-ELT nous a aussi permis de jeter un regard sur les recherches actuelles en astrophysique concernant les cadavres stellaires et les exoplanètes. Ces-dernières, comme nous l'avons évoqué au chapitre \ref{5}, constituent le principal domaine de recherche dans l'astronomie moderne. Cela est d'après moi une très bonne chose, car même si énormément de gens ne s'intéressent guère à leur étude, l'humanité entière est habitée au fond d'elle par la question: sommes-nous seuls dans cet univers ? Il est aussi important de se rappeler que la recherche d'une planète pouvant potentiellement abriter la vie n'est que l'une des premières étapes sur le chemin qui mène jusqu'à la réponse à cette question. Si nous envisagions d'entreprendre le voyage de la Terre à la plus proche des exoplanètes, Proxima b, avec les technologies actuelles, il nous faudrait environ 80000 ans pour atteindre notre objectif; tout cela sans prendre en compte les problèmes de carburant, de nourriture et de survie de l'équipage. Nous voyons ainsi que nous ne sommes qu'au départ d'une longue et merveilleuse quête scientifique et technologique. Celle-ci saura sûrement inspirer les films et les livres de science-fiction d'humanité qui, toujours, garde des étoiles dans les yeux.\smallskip

Cet écrit a aussi répondu à bon nombre des questions qui m'avaient poussées à choisir ce thème comme travail de maturité. En plus de la recherche bibliographique et de la citation des sources, ce travail m'a permis de me familiariser avec le système de composition de document \LaTeX. De plus, l'intérêt que j'ai éprouvé tout au long de la rédaction de ce document m'a donné de bonnes pistes concernant mon future orientation professionnelle.\bigskip

Pour finir, je tiens à remercier tout particulièrement Mme. Romaine Theler pour sa disponibilité et son aide précieuse tout au long de ce travail.





\vfill