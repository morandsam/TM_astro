% !TeX encoding = UTF-8
% !TeX program = xelatex
% !TeX spellcheck = fr
% !TeX root = tm_astro_main.tex



\chapter{Structure interne des étoiles}\label{1}
Dans ce premier chapitre, nous allons tout d'abord traiter de la structure interne des étoiles et la décrire sous une approche mathématique. Par la suite, nous aborderons un modèle stellaire nous permettant de décrire une étoile dans sa globalité.

\section{Equilibre thermodynamique}\label{1.1}

Les étoiles naissent de l'effondrement gravitationnel d'un nuage de gaz interstellaire (§\ref{5}). Au début de leur vie, elles sont principalement composées d'hydrogène qu'elles transformeront progressivement en hélium (§\ref{2}). Cependant, le processus de transformation de l'hydrogène en hélium peut durer des milliards d'années, c'est pourquoi il est intéressant d'étudier la raison pour laquelle une étoile peut conserver cet état pendant une telle durée. En effet, mis à part sa composition chimique, les caractéristiques d'une étoile n'évoluent guère tout au long de la séquence principale\footnotemark[1]; elle se trouve donc dans un certain équilibre qui est qualifié de thermodynamique et de local. La définition d'un équilibre thermodynamique local stipule que les paramètres caractérisant un objet peuvent varier, mais que cette variation est si infime et si lente que nous pouvons donc considérer qu'en chaque point de l'objet, son entourage proche est en équilibre avec lui. Les observables caractérisant cet équilibre sont la pression, la température, la luminosité ainsi que la masse. A chacune de ces observables correspond une équation constituant l'équilibre thermodynamique local.

\subsection{Conservation de la masse}\label{1.1.1}
Cette première équation, appelée équation de continuité de la masse, exprime le principe de conservation de la masse. En effet, la masse d'une étoile reste inchangée durant toute la séquence principale.\smallskip\newline
Tout d'abord, considérons une sphère parfaite\footnotemark[2]; l'aire de celle-ci est définie par:\begin{equation} S=4\pi r^{2}\label{Eq. 1.1}\end{equation}

Ajoutons un élément infinitésimal d'épaisseur \textit{dr} à la surface de la sphère, alors le volume de matière correspond à:\begin{equation} dV=4\pi r^{2}dr\label{Eq. 1.2}\end{equation}

Si nous multiplions \textit{dV} par la masse volumique \textit{$\rho$}, nous avons\textit{ dM(r)=$\rho$(r) dV}; combinons cela avec l'Eq. \ref{Eq. 1.2}:\begin{equation} \boxed{dM(r)=\rho(r)\hspace{2pt}4\pi r^{2}dr}\label{Eq.
	1.3}\end{equation}


\vfill
\footnotetext[1]{La séquence principale, qui sera développée au Chap. \ref{2.1} est la période durant laquelle une étoile fusionne son hydrogène en hélium; cette séquence représente la majeure partie de la vie d'une étoile.}
\footnotetext[2]{Toutes les équations de ce chapitre ont pour hypothèse qu'une étoile est une sphère parfaite sans rotation, ce qui n'est pas exactement vrai; cependant, cette approximation nous suffit dans le cadre de ce travail.}

L'Eq. \ref{Eq. 1.3} représente la conservation de la masse en montrant que celle-ci est proportionnelle au rayon et à des constantes. De plus, l'Eq. \ref{Eq. 1.3} permet aussi d'exprimer la masse d'une étoile en fonction de sa masse volumique.

\subsection{Equilibre hydrostatique}\label{1.1.2}

Si un objet ressemble à une sphère et ne varie pas significativement de taille, c'est soit qu'il est soumis à aucune force, soit qu'il est soumis, dans le cas le plus simple, à deux forces qui se compensent entre elles. Pour les étoiles, c'est la deuxième solution qui est vraie. La première des forces agissant sur une étoile est la force gravitationnelle, elle s'exprime par: \begin{equation} F(r)=\frac{G\hspace{1pt}m\hspace{1pt}dm}{r^{2}}\label{Eq. 1.4}\end{equation}

L'autre force correspond à la force de pression générée par le gaz composant l'étoile.\smallskip

Considérons un cylindre de volume infinitésimal de base \textit{dS} et de hauteur \textit{dr} (Fig. \ref{Fig. A.1}); sa masse vaut:\begin{equation}dm=dr\hspace{1pt}dS\hspace{1pt}\rho\label{Eq. 1.5}\end{equation} 

Avec P étant la force de pression radiative et $\ddot{r}$ l'accélération, la deuxième loi de Newton nous permet
d'écrire\footnotemark[3]:\begin{equation}-\dfrac{G\hspace{1pt}m\hspace{1pt}dm}{r^{2}}-\dfrac{dP(r)}{dr}dr\hspace{1pt}dS=\ddot r\hspace{1pt}dm\label{Eq. 1.6}\end{equation}

Grâce l'Eq. \ref{Eq. 1.5} nous pouvons remplacer \textit{dr\hspace{1pt}dS} dans l'Eq. \ref{Eq. 1.6} et ensuite simplifier le tout par \textit{dm}. De plus, comme l'étoile est en équilibre, nous pouvons affirmer que son accélération est nulle; en réarrangeant les termes nous obtenons:\begin{equation}\boxed{\dfrac{dP(r)}{dr}=-\rho\hspace{2pt}\frac{Gm}{r^{2}}}\label{Eq. 1.7}\end{equation}

En résumé, l'Eq. \ref{Eq. 1.7} nous dit mathématiquement que si la force gravitationnelle et la pression radiative sont opposées et de même intensité, alors l'étoile sera parfaitement équilibrée (Fig. \ref{Fig. A.1}).

\footnotetext[3]{La force de pression agissant sur un élément $dr$ correspond à $P(r)dS-P(r+dr)dS=-\dfrac{dP}{dr}drdS$}

\subsection{Equilibre thermique}\label{1.1.3}

L'étoile ne doit pas seulement être en équilibre hydrostatique pour satisfaire l'équilibre thermodynamique, elle doit aussi compenser les pertes d'énergie qui sont rayonnées sous forme de luminosité, notée $L(r)$. Tout d'abord, nous pouvons exprimer la perte d'énergie d'une couche $dr$ comme étant:\begin{equation}L(r+dr)-L(r)=\dfrac{dL(r)}{dr}dr\label{Eq. 1.8}\end{equation}

Cette perte est compensée par l'énergie provenant des réactions nucléaires. Le taux de libération de l'énergie nucléaire par unité de masse, noté $\epsilon$, appliqué à une sphère se définit comme:\begin{equation}\epsilon\hspace{1pt}dm=\epsilon\rho(r)4\pi r^{2}dr\label{Eq. 1.9}\end{equation}

Si nous égalons l'énergie perdue sous forme de rayonnement avec celle gagnée grâce aux fusions nucléaires, nous obtenons l'équilibre thermique:\begin{equation}\boxed{\dfrac{dL(r)}{dr}=\epsilon\hspace{1pt}\rho(r) 4\pi r^{2}}\label{Eq. 1.10}\end{equation}

L'Eq. \ref{Eq. 1.10} nous permet donc de calculer le taux de variation d'énergie rayonnée, cependant, pour cela, nous devons connaître le taux de libération de l'énergie nucléaire en fonction du rayon. 


\subsection{Transport d'énergie}\label{1.1.4}

De manière générale, il existe trois manières différentes de transporter de l'énergie: la conduction, le rayonnement et la convection. Dans les étoiles, le rayonnement et la convection représentent les deux principaux modes de transport d'énergie. Chacun de ces deux modes possède une équation spécifique décrivant la manière dont il transporte de l'énergie\footnotemark[4]\smallskip

L'équation de transfert d'énergie par radiation se définit comme:\begin{equation}\boxed{\dfrac{dT(r)}{dr}=-\dfrac{3\kappa\rho}{4acT^{3}}\dfrac{L(r)}{4\pi r^{2}}}\label{Eq. 1.11}\end{equation} où $\kappa$ correspond à l'opacité moyenne de l'étoile, $L(r)$ au flux d'écoulement de l'énergie, $T(r)$ à la température et $ac=4\sigma$, avec $\sigma$= constante de Boltzmann.\smallskip

L'équation d'énergie transportée par convection dans les étoiles est donnée par:\begin{equation}\boxed{\dfrac{dT(r)}{dr}= \left(1-\dfrac{1}{\gamma}\right)\dfrac{T}{P}\dfrac{dP}{dr}}\label{Eq. 1.12}\end{equation} avec $\gamma$ représentant l'indice adiabatique du gaz composant l'étoile et $P$ la pression.\bigskip

Ainsi, si une étoile satisfait les critères de la conservation de la masse (Eq. \ref{Eq. 1.3}), de l'équilibre hydrostatique (Eq. \ref{Eq. 1.7}), de l'équilibre thermique (Eq. \ref{Eq. 1.10}) et des transports d'énergie (Eq. \ref{Eq. 1.11} et Eq. \ref{Eq. 1.12}), alors elle se trouve en équilibre thermodynamique. Ces différentes équations différentielles représentent la base d'un modèle stellaire (§\ref{1.2}).

\section{Modèle polytropique}\label{1.2}

Afin de développer un modèle stellaire, il ne suffit pas de ne prendre en considération que la représentation macroscopique de l'étoile (§\ref{1.1}), il faut aussi tenir compte de la description microscopique de celle-ci. Cette description est donnée sous la forme d'une équation d'état. Nous allons considérer dans ce chapitre qu'une étoile est un polytrope, cela signifie que l'équation d'état (Eq. \ref{Eq. 1.13}) la décrivant ne dépend que de la pression $P$ et de la masse volumique $\rho$:\begin{equation}P=K\rho^{\gamma}\label{Eq. 1.13}\end{equation}

où k correspond à une constante arbitraire. Dans un polytrope, l'indice adiabatique $\gamma$ peut aussi être exprimé grâce à l'indice polytropique $n$ de cette manière:\begin{equation}\gamma=1+\dfrac{1}{n}\label{Eq. 1.14}\end{equation} L'indice polytropique permet de simuler une composition particulière d'étoile, cette composition peut être modifiée en fonction de la valeur que nous lui donnons. \bigskip

En réunissant l'Eq. \ref{Eq. 1.3}, l'Eq. \ref{Eq. 1.7}, l'Eq. \ref{Eq. 1.13} et l'Eq. \ref{Eq. 1.14} et après simplifications, nous obtenons:\begin{equation}\dfrac{(n+1)K}{4\pi Gn}\dfrac{1}{r^{2}}\dfrac{d}{dr}\left( \dfrac{r^{2}}{\rho^{\frac{n-1}{n}}}\dfrac{d\rho}{dr}\right) =-\rho\label{Eq. 1.15}\end{equation}

\vfill
\footnotetext[4]{Nous ne démontrerons pas les deux équations de transport d'énergie à cause de leur complexité et leur faible utilité pour ce travail.}




Il est utile pour la suite de la démonstration de remplacer certains termes de l'Eq. \ref{Eq. 1.15} par:
\begin{equation}\rho=\rho_{c} \theta^{n}\hspace{3pt}\label{Eq. 1.16}\end{equation}
\begin{equation}\left[\dfrac{(n+1)K}{4\pi G\rho^{\frac{n-1}{n}}_{c}}\right]=\alpha^{2}\hspace{3pt}\label{Eq. 1.17}\end{equation} 
\begin{equation}r=\alpha\hspace{1pt}\xi\label{Eq. 1.18}\end{equation}

Eq. \ref{Eq. 1.16}: $\rho_{c}$ correspond à la masse volumique au centre de l'étoile, $\theta^{n}$ est une variable sans dimension
comprise entre 0 et 1.\smallskip\newline
Eq. \ref{Eq. 1.17}: La membre de gauche est une constante qui a pour dimension des $m^{2}$, nous la remplaçons par $\alpha^{2}$.\smallskip\newline
Eq. \ref{Eq. 1.18}: Nous pouvons substituer le rayon $r$ par le produit de $\alpha$ (dimension $m$) et de $\xi$ (dimension nulle).\bigskip
Ainsi, en combinant les quatre dernières équations, nous trouvons l'équation de Lane-Emden\footnotemark[5], qui décrit la structure d'une étoile considérée comme une sphère polytropique:\begin{equation}\boxed{\dfrac{1}{\xi^{2}}\dfrac{d}{d\xi}\left( \xi^{2}\dfrac{d\theta}{d\xi}\right) =-\theta^{n}}\label{Eq. 1.19}\end{equation}

Les valeurs de $\xi$ qui résolvent l'Eq. \ref{Eq. 1.19} permettent de calculer des paramètres, comme la masse ou le rayon\footnotemark[6], correspondant à des configurations d'étoiles théoriquement possibles.\smallskip 

Le rayon $R$ d'une étoile polytropique est défini par le premier zéro $\xi_{1}$ de la fonction $\theta(\xi)$\footnotemark[7]:\begin{equation}R=\alpha\hspace{1pt}\xi_{1}\label{Eq. 1.20}\end{equation}

La masse $M$ d'une étoile s'obtient en combinant l'Eq. \ref{Eq. 1.3} et l'Eq. \ref{Eq. 1.20}:\begin{equation}M=\int_{0}^{R}4\pi r^{2}\rho dr=4\pi \alpha^{3} \rho_{c}\int_{0}^{\xi_{1}}\xi^{2}\theta^{n}d\xi\label{Eq. 1.21}\end{equation}

En substituant par l'Eq. \ref{Eq. 1.19} nous obtenons:\begin{equation}M=-4\pi \alpha^{3}\rho_{c}\int_{0}^{\xi_{1}}\dfrac{d}{d\xi}\left(\xi^2\dfrac{d\theta}{d\xi}\right)d\xi=-4\pi \alpha^{3}\rho_{c}\xi^{2}_{1}\left( \dfrac{d\theta}{d\xi}\right)_{\xi_{1}}\label{Eq. 1.22}\end{equation}\bigskip

Afin d'obtenir la valeur de $\xi_{1}$ qui nous permet de calculer le rayon et la masse d'une étoile, il faut résoudre l'équation de Lane-Emden (Eq. \ref{Eq. 1.19}) en fonction de l'indice polytropique $n$ choisi (Fig. \ref{Fig. A.2}). Il est possible de résoudre analytiquement cette équation, pour certaines valeurs de $n$, tandis que pour d'autres nous sommes obligés d'avoir recours à des méthodes numériques.\smallskip 

L'inconvénient du modèle polytropique est qu'il se restreint à une sphère parfaite et sans rotation, ce qui est une bonne approximation, mais qui n'est jamais parfaitement le cas en réalité. De plus, l'indice polytropique $n$ ne permet pas de simuler n'importe quelle composition d'étoile. Il est par contre extrêmement utile, par exemple, pour décrire des étoiles de type naine blanche. 

\vfill
\footnotetext[5]{Cette équation a été nommée en l'honneur des physiciens Jonathan Lane et Robert Emden pour leur travail sur celle-ci.}
\footnotetext[6]{La masse et le rayon ne sont pas les seuls observables pouvant être déterminées par ce modèle, la pression et la densité peuvent l'être aussi, mais nous n'allons pas les traiter ici.}
\footnotetext[7]{Nous pouvons interpréter l'Eq. \ref{Eq. 1.19} comme une équation différentielle ou comme une fonction de $\theta$ par rapport à $\xi$; le premier zéro de la fonction correspond à la première valeur pour laquelle l'équation est résolue.}

