% !TeX encoding = UTF-8
% !TeX program = xelatex
% !TeX spellcheck = fr
% !TeX root = tm_astro_main.tex



\chapter{Structure interne des étoiles}\label{1}
Dans ce premier chapitre, nous allons tout d'abord nous intéresser à la structure interne des étoiles et la décrire sous une approche mathématique. Par la suite, nous aborderons un modèle stellaire nous permettant de décrire une étoile en fonction de sa composition.

\section{Equilibre thermodynamique}\label{1.1}

Les étoiles naissent de l'effondrement gravitationnel d'un nuage de gaz interstellaire (§\ref{4}). Au début de leur vie, elles sont principalement composée d'hydrogène, qu'elles transformeront progressivement en hélium (§\ref{2}). Cependant, le processus de transformation de l'hydrogène en hélium peut durer des milliards d'années, c'est pourquoi il convient d'étudier la raison pour laquelle une étoile peut conserver cet étant pendant une telle durée. En effet, une étoile n'évolue quasiment pas tout au long de la séquence principale\footnotemark[1], mis à part l'augmentation d'hélium et la diminution d'hydrogène; elle se trouve donc dans un certain équilibre.
















\vfill
\footnotetext[1]{La séquence principale, qui sera développée au Chap. \ref{2.1} est la période durant laquelle une étoile fusionne son hydrogène en hélium; cette séquence représente la majeure partie de la vie d'une étoile.}