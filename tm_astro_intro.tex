% !TeX encoding = UTF-8
% !TeX program = xelatex
% !TeX spellcheck = fr
% !TeX root = tm_astro_main.tex


\chapter*{Introduction}

\addcontentsline{toc}{chapter}{Introduction}

\vfill

Comment imaginer ce que 10 milliards d'années peuvent-ils bien représenter ? Durant ce laps de temps, il est possible de vivre environ 125 millions de vies humaines à la suite. Cette durée peut aussi correspondre à 50000 fois la période depuis laquelle l'être humain tel que nous le connaissons est apparu. Néanmoins, il existe bel et bien quelque chose qui soit à l'échelle d'une telle étendue temporelle: la durée de vie du Soleil. En effet, notre étoile est née il y a plus ou moins 4,5 milliards d'années et terminera ses jours dans environ 5,5 milliards d'années, lorsque l'espèce humaine aura probablement déjà disparu depuis très longtemps. Contrairement à ce que nous pourrions penser, les étoiles ne sont pas des objets statiques et inanimés, elles sont en constante évolution et possèdent une vie très agitée. La phase la plus tumultueuse et certainement la plus fascinante à étudier de leur existence est leur mort.\smallskip

 Depuis tout petit déjà, je suis passionné d'astronomie. En grandissant, j'ai toujours cherché à en apprendre plus sur ce fabuleux domaine qui nous concerne tous. Au fur et à mesure des ouvrages de vulgarisation que j'ai pu lire, j'ai ressenti un besoin d'en savoir plus sur certains sujets. Lorsque vînt le temps de choisir mon travail de maturité, j'ai rapidement compris que c'était la parfaite occasion pour creuser plus en profondeur un sujet qui m'intéressait tout particulièrement, la mort des étoiles. Afin d'apporter une problématique moderne à ce sujet théorique nous traiterons aussi du télescope géant européen (E-ELT) ainsi que de ses avancées attendues dans le domaine de l'enrichissement du milieu interstellaire. Après un premier chapitre théorique portant sur la structure interne des étoiles, nous parlerons en détails de l'évolution stellaire. Ensuite, nous focaliserons notre attention sur les explosions mettant fin à la vie des étoiles massives : les supernovas. Dans le chapitre suivant, le télescope géant E-ELT sera présenté sous toutes ses coutures et comparé à d'autres télescopes. Finalement, nous tisserons des liens entre les précédents chapitres en observant d'un point de vue critique les performances attendues par l'E-ELT dans l'étude des cadavres stellaires . 

\vfill
