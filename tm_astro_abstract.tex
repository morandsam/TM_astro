% !TeX encoding = UTF-8
% !TeX program = xelatex
% !TeX spellcheck = fr
% !TeX root = tm_astro_main.tex

\chapter*{Résumé}
\thispagestyle{empty}

\vfill


Malgré leur apparente quiétude les étoiles sont le siège de perpétuelles réactions thermonucléaires modifiant constamment leur composition. Lorsque la fusion nucléaire s'arrête, faut de combustible, l'engrenage menant l'étoile tout droit à sa mort se met en route. En fonction de leur masse initiale, les étoiles subissent différents processus mettant fin à leur vie. De manière générale, plus la masse d'une étoile est élevée, plus sa mort sera cataclysmique. Les explosions de supernovas, qui mettent fin à la vie des étoiles pesant au minimum 9$M_\odot$, représentent tout à fait ce que nous pouvons nous imaginer lorsque nous pensons à un phénomène dévastateur. Ces explosions éparpillent dans le milieu interstellaire toutes les couches de l'étoile, sauf son noyau. Au fil du temps, lorsque suffisamment de poussières d'étoiles provenant de précédentes morts d'étoiles se réunissent sous l'effet de la gravité, une nouvelle génération d'étoiles sera bientôt prête à voir le jour. Lors de la naissance d'une étoile, il y a de fortes chances que quelques exoplanètes se forment autour d'elle. Ces-dernières pourront prochainement être directement détectées et caractérisées par le télescope géant européen (E-ELT), qui sera, avec son miroir de 39  mètres de diamètre, le plus grand télescope jamais construit lors de sa mise en service prévue en 2024. Il sera ainsi en mesure de découvrir des planètes plus ou moins semblables à la Terre, de part leur composition chimique ou leur température par exemple. Cela nous permettrait donc d'avancer quelque peu dans notre quête de la découverte d'une potentielle vie extraterrestre.




\vfill



